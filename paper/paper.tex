% This is "www2009-sample.tex" copied from "www2005-sample.tex" V1.2 January 26 2004
% This file should be compiled with V1.4 of "www2009-submission.class"
%
% This example file demonstrates the use of the 'www2009-submission.cls'
% V1.4 LaTeX2e document class file. It is for those submitting
% articles to the WWW'04 Conference WHO DO NOT WISH TO 
% STRICTLY ADHERE TO THE SIGS (PUBS-BOARD-ENDORSED) STYLE.
% The 'www2009-submission.cls' file will produce a similar-looking,
% albeit, 'tighter' paper resulting in, invariably, fewer pages.
%
% ----------------------------------------------------------------------------------------------------------------
% This .tex file (and associated .cls V1.4) produces:
%       1) NO Permission Statement
%       2) WWW'04-specific conference (location) information
%       3) The Copyright Line with ACM data
%       4) NO page numbers
%
% ---------------------------------------------------------------------------------------------------------------
% This .tex source is an example which *does* use
% the .bib file (from which the .bbl file % is produced).
% REMEMBER HOWEVER: After having produced the .bbl file,
% and prior to final submission, you *NEED* to 'insert'
% your .bbl file into your source .tex file so as to provide
% ONE 'self-contained' source file.
%
% ================= IF YOU HAVE QUESTIONS =======================
% Questions regarding the SIGS styles, SIGS policies and
% procedures, Conferences etc. should be sent to
% Julie Goetz (goetz@acm.org) or Adrienne Griscti (griscti@acm.org)
%
% Technical questions only to
% Gerald Murray (murray@acm.org)
% ===============================================================
%
% For tracking purposes - this is V1.2 - January 26 2004
\documentclass{www2009-submission}


\begin{document}
%
\title{Webnnel: A Channel-based Multimodal Web Navigation System}

%\subtitle{[Extended Abstract]
%\titlenote{A full version of this paper is available as
%\textit{Author's Guide to Preparing ACM SIG Proceedings Using
%\LaTeX$2_\epsilon$\ and BibTeX} at
%\texttt{www.acm.org/eaddress.htm}}}
%
% You need the command \numberofauthors to handle the "boxing"
% and alignment of the authors under the title, and to add
% a section for authors number 4 through n.
%
% Up to the first three authors are aligned under the title;
% use the \alignauthor commands below to handle those names
% and affiliations. Add names, affiliations, addresses for
% additional authors as the argument to \additionalauthors;
% these will be set for you without further effort on your
% part as the last section in the body of your article BEFORE
% References or any Appendices.

\numberofauthors{6} %  in this sample file, there are a *total*
% of EIGHT authors. SIX appear on the 'first-page' (for formatting
% reasons) and the remaining two appear in the \additionalauthors section.
%
\author{
% You can go ahead and credit any number of authors here,
% e.g. one 'row of three' or two rows (consisting of one row of three
% and a second row of one, two or three). 
%
% The command \alignauthor (no curly braces needed) should
% precede each author name, affiliation/snail-mail address and
% e-mail address. Additionally, tag each line of
% affiliation/address with \affaddr, and tag the
% e-mail address with \email.
%
% 1st. author
\alignauthor
      Oshani Seneviratne\\
       \affaddr{CSAIL, MIT}\\
       \email{oshani@csail.mit.edu}
% 2nd. author
\alignauthor
      Chen-Hsian Yu\\
       \affaddr{CSAIL, MIT}\\
       \email{chyu@csail.mit.edu}
% 3rd. author
\alignauthor 
      Randall Davis\\
       \affaddr{CSAIL, MIT}\\
       \email{davis@csail.mit.edu}
\maketitle

\begin{abstract}

With the success of Web technologies, people are seeking similar experiences gleaned from other domains to be applied to web navigation. Based on this trend, we can envision an application using Web browser to navigate information on a big display in the home environment. Webnnel is an implementation of a Web navigation system to represent web sites as TV channels and allow users to use multi-modalities to control navigation. Currently, Webnnel contains three input modalities to demonstrate the interactive behaviors, including speech recognizer, (mouse) gesture and keyboard input. Furthermore, Webnnel�s architecture is flexible to connect with other kinds of input modalities to provide better interaction on Web navigation.

\end{abstract}

% A category with only the three required fields
\category{1.2}{Artificial Intelligence}{Law}
\category{H.3.4}{Information Systems}{User Profiles and Alert Services}
\category{H.3.5}{Information Systems}{Web-Based Services}
%A category including the fourth, optional field follows...
\category{K.5}{Computing Milieux}{Legal Aspects of Computing}

\terms{}

\keywords{Web Crawling, Semantic Web, Creative Commons Rights Expression Language, Accountability in RDF, Semantic Web Reasoning }

\section{Introduction}

Before the era of the Web, people used to use specific software to satisfy the need. However, the Web has become an important medium for delivering information nowadays. There are more and more people relying on it for work and entertainment, such as checking e-mails, reading news, watching videos, listening to music and shopping on the Web. Users even seem to have their free operating system on the Web [13,14].
In this paper, we envision a future application to use Web browser on a big screen TV in the home environment. We have designed a Web navigation system to represent web sites as TV channels and allow users to use multiple modalities to navigate them, even to control and change the content of the web sites. In short, the user can use the input modalities to send out the request to the system, and the system would respond correspondingly. (Figure 1)

\begin{figure*}[h]
  \centerline{\epsfig{file=images/design.jpg,  width=1\linewidth}}
  \caption{System Design}
  \label{fig-design}
\end{figure*}

\section{Background on the Creative Commons}

Creative Commons (CC), has tried to answer questions pertaining to the promotion of reasonable and flexible copyright regime for the World Wide Web for several years now. One of the widely accepted solutions so far has been
to create licenses that permit sharing and reuse with conditions, clearly communicated in human readable form. The other option is to leverage digital networks themselves to make licensed work more reusable and  \cite{hal08cc}

\section{Motivating Scenario}

User Study - Harvest Some URIs and run the system over with all those

Focus this more as a validator

Science Commons work, where people have so many different sources, and they might want to run their own work on the validator to see any license violations of works.

Alice is an avid Flickr user and she uploads her photos to her account regularly. In her Flickr account settings she has applied "CC-BY-3.0" to all her photos by default. This means she allows anybody to 
Bob sees one of her photos which interest him, and he 
embeds the photo in his 

\section{Related Work}

DRM - Joan Feiganbaum's work


Much of the work that has been done in this area includes incorporating the CC license along with the meta data for an image. LibLicense ~/cite{liblicense} provides a low-level license metadata integration for applications. 

XMP

\section{System Design}

Use of the AIR reasoner for more complicated license scenarios like original work licensed under CC-BY-SA being licensed with ACM license and CC-BY.

\begin{figure*}[h]
  \centerline{\epsfig{file=images/design.jpg,  width=1\linewidth}}
  \caption{System Design}
  \label{fig-design}
\end{figure*}

This system has 4 major components.

The crawler will look at a given site and determine if there are any embedded Flickr photos.
If such photos are detected, License Checker will determine whether it is under a Creative Commons license.
The User Checker will find out other identifying information related to the original creator of the photo. Since all Creative Commons licensed works should give attribution to the original creator, the crawler will again check whether the name or any other identifying information of the original creator appears on the page the photo is embedded on.
The notifier will send a notification to the original creator about the data usage and license terms violation.

The crawler implements a basic BFS algorithm to check for Flickr image URIs in a given site.

A Flickr image URI takes one of the following formats:

\begin{figure}
  \centerline{\epsfig{file=images/uri_format.png,  width=1\linewidth}}
  \caption{Format of Flickr Image URIs}
  \label{fig-flickr}
\end{figure}


From the photo URI, the id of the photo can be extracted. Using this id, all the information related to the photo could be obtained by calling several methods in the Flickr API. This information also includes the original creator's Flickr user account, name and CC license information pertaining to the photo. Again, using the crawler, the page is checked to see whether the original creator is attributed (only if the photo has a CC license attached). With the QDOS SPARQL endpoint, more of the photo owner's data (FOAF URI, etc) could be obtained, to perform a thorough search and notify in case of a license terms violation.


\section{Applicability to other types of Licenses}
\cite{berners08n3}

\section{Issues}

This will not work if the images are downloaded from Flickr and embedded in the site.
Can write license information as EXIF data, but it could be easily overwritten
BFS is only limited to all the links within the seed site
Locality of the search for creator within the web page should be improved
Notification to user for any license terms violation is not implemented. Needs user consent and worry about DPAs, etc.

\section{Future Work}

Extend to other CC licenses
Extend to other data usage scenarios (for e.g. YouTube)
Track provenance of images using metadata (instead of relying on the URIs)
Social Verification: i.e. use the FOAF graph to control access for viewing, tagging and commenting on photo sharing sites
Automatically inject the attribution details whenever a photo is linked

\section{Conclusions}

%ACKNOWLEDGMENTS are optional
\section{Acknowledgments}
This work was carried out for Web Science Research Initiative (WSRI) Exchange funded by the UK Engineering and Physical Sciences Research Council (EPSRC) under grant number EP/F013604/1.
We would also like to thank our colleague Harith Alani, Steve Harris and various devs on IRC channel 'cc' who helped us with numerous suggestions to explore. Last, but not least, we would also like to thank Mr. 
Puneet Kishore who contributed to us formulating the problem and coming up with a potential solution (he complained when one of us inadvertently published one of his CC licensed photos).

%
% The following two commands are all you need in the
% initial runs of your .tex file to
% produce the bibliography for the citations in your paper.
\bibliographystyle{abbrv}
\bibliography{references}  % sigproc.bib is the name of the Bibliography in this case
% You must have a proper ".bib" file
%  and remember to run:
% latex bibtex latex latex
% to resolve all references
%
% ACM needs 'a single self-contained file'!
%
%APPENDICES are optional
%\balancecolumns



% That's all folks!
\end{document}
